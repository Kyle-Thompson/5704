
% ===========================================================================
% Title:
% ---------------------------------------------------------------------------
% to create Type I fonts type "dvips -P cmz -t letter <filename>"
% ===========================================================================
\documentclass[11pt]{article}       %--- LATEX 2e base
\usepackage{latexsym}               %--- LATEX 2e base
%---------------- Wide format -----------------------------------------------
\textwidth=6in \textheight=9in \oddsidemargin=0.25in
\evensidemargin=0.25in \topmargin=-0.5in
%--------------- Def., Theorem, Proof, etc. ---------------------------------
\newtheorem{definition}{Definition}
\newtheorem{theorem}{Theorem}
\newtheorem{lemma}{Lemma}
\newtheorem{corollary}{Corollary}
\newtheorem{property}{Property}
\newtheorem{observation}{Observation}
\newtheorem{fact}{Fact}
\newenvironment{proof}           {\noindent{\bf Proof.} }%
                                 {\null\hfill$\Box$\par\medskip}
%--------------- Algorithm --------------------------------------------------
\newtheorem{algX}{Algorithm}
\newenvironment{algorithm}       {\begin{algX}\begin{em}}%
                                 {\par\noindent --- End of Algorithm ---
                                 \end{em}\end{algX}}
\newcommand{\step}[2]            {\begin{list}{}
                                  {  \setlength{\topsep}{0cm}
                                     \setlength{\partopsep}{0cm}
                                     \setlength{\leftmargin}{0.8cm}
                                     \setlength{\labelwidth}{0.7cm}
                                     \setlength{\labelsep}{0.1cm}    }
                                  \item[#1]#2    \end{list}}
                                 % usage: \begin{algorithm} \label{xyz}
                                 %        ... \step{(1)}{...} ...
                                 %        \end{algorithm}
%--------------- Figures ----------------------------------------------------
\usepackage{graphicx}

\newcommand{\includeFig}[3]      {\begin{figure}[htb] \begin{center}
                                 \includegraphics
                                 [width=4in,keepaspectratio] %comment this line to disable scaling
                                 {#2}\caption{\label{#1}#3} \end{center} \end{figure}}
                                 % usage: \includeFig{label}{file}{caption}


% ===========================================================================
\begin{document}
% ===========================================================================

% ############################################################################
% Title
% ############################################################################

\title{LITERATURE REVIEW:  Achieving Lock-Free Writes and Wait-Free Reads with Persistence}


% ############################################################################
% Author(s) (no blank lines !)
\author{
% ############################################################################
Kyle Thompson\\
School of Computer Science\\
Carleton University\\
Ottawa, Canada K1S 5B6\\
{\em kylejthompson@cmail.carleton.ca}
% ############################################################################
} % end-authors
% ############################################################################

\maketitle



% ############################################################################
\section{Introduction} \label{intro}
% ############################################################################

As soon as parallel computing became more than an idea, a question had to be answered. How to
get multiple threads to work on the same data? The solution at the time was to use locks and mutexes 
[CITATION]to prevent multiple threads from writing to the data, or reading from it while another is writing 
to it. While this works it is not without its issues. For instance in this model threads can enter deadlock 
[CITATION] by which they have a lock and are waiting for another which is currently possessed by a 
different thread that is waiting on the original threads lock. In this situation the threads will be stuck and 
simply wait until the process is terminated.

Enter lock-free programming. This new paradigm is not one where, as the name might suggest,
a thread never has to wait or that locks are not used, but rather one that simply guarantees
progress in the program by at least one thread [CITATION]. The problem though is that lock-free 
programming is usually very hard and error prone so despite its benefits, it is rarely used.

The goal of this paper then is to demonstrate a method for which any pointer based data structure
can be easily made lock-free for writes and wait-free for reads by making the data structure persistent
and leveraging the immutability of the structure at a given timestamp.


% ############################################################################
\section{Literature Review} \label{litrev}
% ############################################################################

Give an overview of the relevant literature. Cite all relevant
papers, like \cite{DEL07}, \cite{PD07}, \cite{DER07}, \cite{LDR07},
\cite{DLX06}, \cite{CDE06}, and \cite{DFL06}. Outline for each paper
the relevant results in relation to your project. Make sure that you
don't just list all relevant papers in random order. Devise a scheme
to group papers by subject. The goal is to present to the reader the
state-of-the-art in the field selected for your project.


% ############################################################################
% Bibliography
% ############################################################################
\bibliographystyle{plain}
\bibliography{my-bibliography}     %loads my-bibliography.bib

% ============================================================================
\end{document}
% ============================================================================
